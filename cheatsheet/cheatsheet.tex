\documentclass[papersize=a4,paper=landscape,11pt]{scrartcl}

\usepackage[top=0.5cm, bottom=2.5cm, left=0.5cm, right=0.5cm]{geometry}

\usepackage[default]{sourcesanspro}
\usepackage[utf8]{inputenc}
\usepackage[T1]{fontenc}
\usepackage{lipsum}
\usepackage{multicol}

\usepackage{tabularx}

\usepackage[table]{xcolor}
\definecolor{Primary}{HTML}{6200EE}
\definecolor{Gray}{gray}{0.85}
\definecolor{LightPrimary}{gray}{0.95}

\newcommand{\tableHeaderThree}[3]{\rowcolor{Primary} \leavevmode\color{white}{\bfseries #1} & \leavevmode\color{white}{\bfseries #2} & \leavevmode\color{white}{\bfseries #3}}
\newcommand{\tableHeaderTwo}[2]{\rowcolor{Primary} \leavevmode\color{white}{\bfseries #1} & \leavevmode\color{white}{\bfseries #2}}
\newcommand{\tableHeaderOne}[1]{\rowcolor{Primary} \leavevmode\color{white}{\bfseries #1}}
\newcommand{\oddRow}{\rowcolor{LightPrimary}}
\newcommand{\evenRow}{\rowcolor{Gray}}

\definecolor{pblue}{rgb}{0.13,0.13,1}
\definecolor{pgreen}{rgb}{0,0.5,0}
\definecolor{pred}{rgb}{0.9,0,0}
\definecolor{pgrey}{rgb}{0.46,0.45,0.48}

\usepackage{hyperref}

\usepackage{listings}
\lstset{language=Java,
	showspaces=false,
	showtabs=false,
	breaklines=true,
	showstringspaces=false,
	breakatwhitespace=true,
	commentstyle=\color{pgreen},
	keywordstyle=\color{pblue},
	stringstyle=\color{pred},
	basicstyle=\ttfamily,
	numbers=none,
	frame=single,
	rulecolor=\color{Gray}
}

\setlength{\columnsep}{0.5cm}

\usepackage[english]{isodate}

\title{Java Cheat Sheet}
\author{Simon Buttgereit \and Markus Theil}
\date{Version: \isodate\today}

\begin{document}
\begin{multicols*}{3}
\maketitle
\subsection*{Numerical Data Types}
\begin{tabularx}{\columnwidth}{llX}
	\tableHeaderThree{Data Type}{Size}{Range/Example}\\
	\oddRow byte & 8 bit & $-128 \ldots 127$\\
	\evenRow short & 16 bit & $-32768\ldots32767$\\
	\oddRow int & 32 bit & $-2^{31}\ldots2^{31}-1$\\
	\evenRow long & 64 bit & $-2^{63}\ldots2^{63}-1$\\
	\oddRow float & 32 bit & $-0.1$f\\
	\evenRow double & 64 bit & $0.3$\\
\end{tabularx}

\subsection*{Other Data Types}
\begin{tabularx}{\columnwidth}{lX}
	\tableHeaderTwo{Data Type}{Range/Example}\\
	\oddRow boolean & true, false\\
	\evenRow char & 'c' (single character)\\
	\oddRow String & "I'm a string"\\
\end{tabularx}

\subsubsection*{Variables}
Use variables to read and store values which are changing throughout the execution.
\begin{lstlisting}
int i;
int j = 3;
i = j + 27;
\end{lstlisting}
\subsection*{Arrays}
    Use arrays to group multiple identically typed values together. Index range goes from  0 to length - 1.
\begin{lstlisting}
int[] myArray = new int[10];
myArray[2] = 5;

// a 2D array or matrix
int[][] intArray = new int[10][20];

// a 3D array
int[][][] intArray = null;
intArray = new int[10][20][10];
\end{lstlisting}

\subsubsection*{Length/Size of Arrays and Strings}
The index range of arrays and strings goes from 0 to length - 1.
An array has a length property, whereas a String object has a length method.

\begin{lstlisting}
int[] myArray = new int[50];
int arrayLength = myArray.length;
String s = "I'm a string!";
int stringLength = s.length();
\end{lstlisting}

\subsection*{Conditional Statements}
Use the if statement, if you want to execute code only if some boolean conditions hold or not.
\begin{lstlisting}
if(statement) {
  ...
} else if(statement) {
  ...
} else {
  ...
}
\end{lstlisting}
Use the switch statement, if you have a multitude of possible execution paths.
\begin{lstlisting}
switch(variable) {
  case 1:
    System.out.println("A");
    break;
  case 5:
    System.out.println("B");
    break;
  default:
    System.out.println("Otherwise");
    break;
}
\end{lstlisting}

\subsection*{Loop Statements}
Below you can find examples of different loop statements, each printing the numbers 0 to 9.
\begin{lstlisting}
for(int i = 0; i < 10; i++) {
  System.out.println("Number " + i);
}
\end{lstlisting}

\begin{lstlisting}
int i = 0;
while(i < 10) {
  System.out.println("Number " + i);
  i++;
}
\end{lstlisting}

\begin{lstlisting}
int i = 0;
do {
  System.out.println("Number " + i);
  i++;
} while(i < 10);
\end{lstlisting}

\subsection*{Methods}
Use methods to repeatedly use identical code.
\begin{lstlisting}
static public int addThree(int i) {
  int j = i + 3;
  System.out.println(i + " -> " + j);
  return j;
}
\end{lstlisting}

\subsection*{Comments}
Use comments in order to add textual annotations to your code or temporarily disable some code path.
\begin{lstlisting}
// single-line comment

/*
 * multi-line comment
 */
\end{lstlisting}

\subsection*{Arithmetic Operators}
Use arithmetic operators for numerical calculations.
\begin{tabularx}{\columnwidth}{lX}
	\tableHeaderTwo{Operator}{Explanation}\\
	\oddRow + & Addition\\
	\evenRow - & Subtraction\\
	\oddRow * & Multiplication\\
	\evenRow / & Division\\
	\oddRow \% & Division Remainder (Modulus)\\
	\evenRow ++ & Increment\\
    \oddRow -{}- & Decrement\\
\end{tabularx}

\subsection*{Relational Operators}
\begin{tabularx}{\columnwidth}{lX}
	\tableHeaderTwo{Operator}{Explanation}\\
	\oddRow == & equal\\
	\evenRow != & non-equal\\
	\oddRow > & greater\\
	\evenRow < & less\\
	\oddRow >= & greater or equal\\
	\evenRow <= & less or equal\\
\end{tabularx}

\subsection*{Logical Operators}
\begin{tabularx}{\columnwidth}{lX}
	\tableHeaderTwo{Operator}{Explanation}\\
	\oddRow \&\& & AND (short-circuit)\\
	\evenRow || & OR (short-circuit)\\
	\oddRow ! & NOT\\
\end{tabularx}

\subsection*{Bitwise Operators}
\begin{tabularx}{\columnwidth}{lX}
	\tableHeaderTwo{Operator}{Explanation}\\
	\oddRow \& & AND\\
	\evenRow | & OR\\
	\oddRow \textasciicircum & XOR\\
	\evenRow \textasciitilde & NOT\\
	\oddRow \textgreater{}\textgreater & shift right\\
	\evenRow \textless{}\textless & shift left\\
	\oddRow >{}>{}> & shift right zero fill\\
\end{tabularx}


\subsection*{IOUtils Package}
You can find a pre-built package at \url{https://bit.ly/2M2XXHH}.
The source code is at \url{https://github.com/telematik-tu-ilmenau/AuPUtils}.

\begin{lstlisting}
import aup.*;
...
int n = IOUtils.readInt();
System.out.println("Number: " + n);
\end{lstlisting}

\subsubsection*{Random Numbers}
Math.random returns a random number in $[0,1)$.
\begin{lstlisting}
double rD = Math.random() * 5 + 1;
...
int r = (int)(Math.random() * 5 + 1);
\end{lstlisting}
Java also ships a separate class for generating random numbers.
\begin{lstlisting}
Random rand = new Random();
int r = rand.nextInt(5);
\end{lstlisting}

\subsection*{Mathematical Functions}
\begin{tabularx}{\columnwidth}{lX}
	\tableHeaderTwo{Method}{Explanation}\\
	\oddRow Math.abs(x) & |x|\\
    \evenRow Math.asin(x) & $\sin^{-1}(x)$\\
	\oddRow Math.acos(x) & $\cos^{-1}(x)$\\
	\evenRow Math.atan(x) & $\tan^{-1}(x)$\\
	\oddRow Math.sin(x) & $\sin(x)$\\
	\evenRow Math.cos(x) & $\cos(x)$\\
	\oddRow Math.tan(x) & $\tan(x)$\\
	\evenRow Math.log(x) & $\log(x)$\\
	\oddRow Math.sqrt(x) & $\sqrt{x}$\\
	\evenRow Math.exp(x) & $e^x$\\
	\oddRow Math.pow(x,y) & $x^y$\\
	\evenRow Math.floor(x) & $\lfloor x \rfloor$\\
	\oddRow Math.ceil(x) & $\lceil x \rceil$
\end{tabularx}

\subsubsection*{Java Documentation}
Find the latest Java library documentation at \url{https://docs.oracle.com/javase/10/docs/api/overview-summary.html}.
\end{multicols*}

\end{document}
